\section{简介}

如图~\ref{fig:overview},以C++为基本,对基础部分进行学习。
了解C++基本语法,学习类与数据对象,容器和算法,\textbf{面向对象和泛型编程(重点)},编译与底层,C++新特性。

\begin{figure}[]
	\centering
	\includegraphics[width=0.5\columnwidth]{pic/overview.png}
	\caption{基础部分学习总览}
	\label{fig:overview}
	% \vspace{-5pt}
\end{figure}

\section{C++ Review}
C++语言同时支持五种编程风格:C风格(面向过程)、基于对象、面向对象、泛型和函数式。
在C++11之前抽象存在若干的缺陷,最严重的是缺少自动内存管理和对象级别的消息发送机制。
现代C++语言可以看作是三部分组成的:1.低级语言,大部分继承自C;2.现代高级语言特性,允许我们定义自己的类型以及组织大规模程序和系统;3.标准库,利用高级特性来提供有用的数据结构和算法。
\begin{tcolorbox}[boxsep=-0.05in]
	\underline{C++和C有什么区别:} C++有与C一样的低级语言特性,面向过程,这部分主要继承于C;同时C++,尤其是C++11拥有众多的高级语言特性,使得程序更加易于开发,提供非常多的标准库,使得实现特定的数据结构和算法更加简单。同时,C++相比于C,它支持更多的编程模式,比如面向对象,基于对象,函数式编程以及泛型编程。
	\textbf{参考答案:}设计思想上,C++是面向对象的语言,而C是面向过程的结构化编程语言;语法上,C++具有封装、继承和多态三种特性,C++相比C,增加多许多类型安全的功能,比如强制类型转换、C++支持范式编程,比如模板类、函数模板等。
\end{tcolorbox}

\subsection{C++基础}
\begin{itemize}
	\item \textbf{输入输出流:} iostream,$cin >>$, $cout <<$。
	\item \textbf{控制流:} while(condition) {statement} ; for(int i=1;i<=10;++i){statement} ; if(condition){statement};
	\item \textbf{基本内置类型:} 算术类型(整形(包括字符和布尔)|浮点型):bool, char, wchar\_r, char16\_t, char32\_t, short, int, long, long long, float, double, long double。 bit(1)-byte(8)-word(32/64) byte是寻址的最小内存单元。无符号数永远不会小于零,切勿混用。'a'单引号是char型字面值,"hello"叫字符串字面值,字符串型字面值结尾会补一个空字符。
	\item \textbf{初始化和赋值}要加以区分,但事实上无关紧要。赋值是将对象的当前值擦除,而以新值代替。建议初始化所有的内置类型的变量。
	\begin{lstlisting}[caption={}]
		int units = 0;
		int units = {0};
		在C++11中用花括号初始化变量得到了全面的应用,该形式称为列表初始化。该方式在赋值中如何会
		丢失值(非强制类型转换),编译器会报错。
		int units{0}; 
		int units(0);
	\end{lstlisting}
	\item \textbf{extern} 为了支持分离式编译,C++将声明和定义区分开来。声明只是让名字为程序所知,而定义负责创建与名字关联的实体。在实际执行上,定义申请存储空间,也可能会赋初值,而声明不会。 extern int i; int j; 但是extern int i = 1;是声明,因为包含了显示初始化。变量能且只能被定义一次,但是可以被多次声明。主要用来解决多个文件中使用同一个变量的情况。
	\item \textbf{复合类型}指基于其他类型定义的类型,如引用和指针。一般情况下说引用就是指左值引用,而在C++11中新增了右值引用。类型修饰符放在名字之前,而不是数据类型一起不容易产生误导。
	\item \textbf{引用} int ival = 1024; int $\&$ref = ival;引用必须被初始化。引用即别名,且只能绑定一次,相当于又起了一个名字。注意,不能定义引用的引用,因为应用本身不是一个对象。
	\item \textbf{指针} int ival = 42; int * p = \&ival; \&取地址符。 $cout << *p$; *是访问指针所指的对象,叫解引用符。
	\item \textbf{空指针} 不指向任何对象,因此编程中在使用一个指针之前要先判断其是否为空。nullptr(C++11) NULL 0; NULL叫做预处理变量,在cstdlib中定义,它的值就是0。
	\item \textbf{void*指针} 可用于存放任意对象的地址。但是void*不记录对象类型,因此不能用其操作/访问对象,只能比较地址或者作为返回值(传递)。
	\item \textbf{指向指针的指针} 指针与引用不同,是内存中的对象。因此可以有指向指针的指针。int **ppi = \&pi;同时可以有指向指针的引用; int* \&r = p; int* 说明引用的类型,而\&说明它的本质是引用,此时r就是p的一个别名,直接和p一样用就可。遵循从右向左解释。
	\item \textbf{const限定符} const int bufSize = 512; 缓冲区大小,且不能改变,只能访问内容,不能修改。在编译过程中,编译器会将所有的bufSize都直接换成512。为了避免对同一变量的重复定义,默认情况下,const对象被设定为仅在文件内有效。当多个文件中出现同名的const变量时,其实是不同的独立定义了变量。当有必要在文件之间共享,而不是编译器为每个文件生成独立的变量时候,只需要在一个文件中定义const,而在多个文件中声明。全部加上extern关键字,只在一个地方初始化即可。
	\item \textbf{const的引用} 可以引用,但是不能通过引用修改对象,因此它还是一个常量。const int i = 1; const int \& ri = i;
	\item \textbf{常量引用的非常量对象} int i = 42; const int \&r1 = i; const int \&r2 = 42; const int \&r3 = r1*2; 总之引用的限制条件必须大于右边进行初始化的量。
	\item \textbf{指向常量的指针} 令指针指向常量或非常量。 const double * pi = 3.14。类似的,左边的限制条件必须大于右边。
	\item \textbf{常量指针} int *const p; 因为指针本身是对象,因此允许将指针本身定义为常量,不变的是指针的值,而非指向的那个值。\textbf{顶层const} top-level const表明指针本身是个常量,\textbf{底层const} low-level表明指针所指的对象是一个常量。
	\item \textbf{常量表达式}指在编译阶段就能知道计算结果的表达式。c++11中运行将变量声明为constexpr类型提示编译器。constexpr int mf = 20;对于类型比较简单,值也显而易见的称为字面值类型。
	\item \textbf{typedef} typedef double *p; p 就是double*; 在C++11中定义了一个新方法:using SI = Sales\_item; 当对指针时候,注意区分指向常量的指针和常量指针。
	\item \textbf{auto} auto item = v1 + v2; auto定义的变量必须有初始值。在一行中定义多个变量用auto时需要注意,因为auto只有一个,当初始值赋值可能判断为不同的类型,会报错。
	\item \textbf{decltype} decltype(f()) sum = x; 判断f()返回的类型,作为sum的类型。当涉及到引用的时候还挺麻烦。如decltype((i))返回的就是一个引用类型。
	\item \textbf{struct自定义数据类型} struct Sales{std::string bookNo; unsigned units = 0; double revenue = 0.0;};
	\item \textbf{头文件} 像是一个声明,类通常被定义在头文件中,类所在的头文件的名字应与类的名字一样。头文件通常包含那些只能被定义一次的实体,如类、const、和constexpr变量。
	\item \textbf{头文件保护符} 为了不多次包含相同的头文件,通常需要$\#$ifndef SALES\_DATA\_H $\#$define SALES\_DATA\_H $\#$include <string> $\#$endif。预处理变量无视作用域规则。	
\end{itemize}

\subsection{字符串、向量和数组}
上一小节介绍的是内置类型,这一节是抽象数据类型库。其中string和vector是两种最重要的标准库类型,string支持可变长字符串,vector支持可变长的集合。

\textbf{using} using std::cin; using namespace::std; 头文件不应该包含using声明,防止被放在不同文件中出错。

\subsubsection{String}
string对象上的操作:
\begin{itemize}
	\item \textbf{$os << s$} 将s写到输出流os当中,返回os. 注意输入输出都以空格结束,即一有空格就算作下一个字符串了。
	\item \textbf{$is >> s$} 从is中读取字符串付给是,字符串以空白分割。
	\item \textbf{getline(is,s)} 从is读取一行付给s,返回is。
	\item \textbf{s.empty()}是为空返回true,否则返回false。注意这里所有的返回值都是string::size\_type,不算int啥的,因此最好用 decltype(s.size()) n = s.size()
	\item \textbf{s.size()}返回s中字符的个数。
	\item \textbf{s[n]}返回s中第n个字符的\underline{引用},位置n从0记起。有点儿像切片
	\item \textbf{s1+s2}字符串相连。
	\item \textbf{s1=s2}以s2替换s1。
	\item \textbf{s1==s2}判断相等,相等性对大小写敏感。对于比较,则利用字符在字典中的顺序,对大小写敏感
	\item \textbf{cctype}处理string对象中的字符,可以用到cctype头文件,判断大小写,标点符号,十六进制,是不是空格,可不可以打印,字母还是数字。注意,尽可能的使用C++版本的头文件,前面多个c,且去掉了.h后缀。
	\item \textbf{对于输出字符串中的每一个字符} $for(auto c : str) cout << c <<endl;$注意如果向改变str中的值,必须把c定义成引用。当用引用时,这里其实是将c一次绑定到str上的每个字符,操作c就是操作了str,否则就无法改变str。
	\item \textbf{下标运算符} str[1] 返回的是对位置1的引用。访问字符串之前,最好先.empty()一下。
\end{itemize}

\subsubsection{Vector}
vector是一个类模板,对于类模板需要将指定模板实例化成什么样的类,需要提供哪些信息由模板决定。提供信息的方式是尖括号,如$ vector<int> ivec; vector<vector<int>> ivvec; $
\begin{itemize}
	\item \textbf{初始化}vector<int> ivec(10, -1);  还有列表初始化等。
	\item \textbf{向vector对象中添加元素} v.push\_back(i); 这里vector额外提供了方法,允许我们进一步提升动态添加元素的性能。
	\item \textbf{循环体内部有向vector添加元素的语句,则不能使用for循环} 这是因为vector是动态变化的,内存不够用的时候会自动开辟新内存,导致其下标指向的元素也会随着vector添加元素的数量动态变化。
	\item \textbf{v.empty()}
	\item \textbf{v.size()}
	\item \textbf{v[n]}
	\item \textbf{v1 = v2}
	\item \textbf{v1 = {a,b,c...}}
	\item \textbf{v1 == v2}
	\item $vector<int>::size\_type$
	\item \textbf{不能用下标形式添加元素} 因为vec里其实是不包含任何元素的。只能push\_back()
\end{itemize}

\subsubsection{Iterator迭代器}
除去下标方式,也可以用迭代器的方式。所有标准库容器都可以使用迭代器。对于有迭代器的类型,同时就拥有返回迭代器的成员。 auto b = v.begin(); auto e = v.end(); end()一般用来判断迭代器是否到头。同时对于不需要写的操作,也定义有v.cbegin(),返回const\_iterator类型的对象。

迭代器的运算符:
\begin{itemize}
	\item \textbf{*iter} 返回所指元素的引用。(不是引用就没办法对其进行修改)
	\item \textbf{iter->mem} 获取该元素的mem成员,等价于(*iter).mem
	\item \textbf{++iter}
	\item \textbf{--iter}
	\item \textbf{iter1==iter2}
	\item \textbf{迭代器类型}一般是iterator或者const\_iterator
	\item \textbf{迭代器解引用} 可获得迭代器所指的对象,如果该对象的类型恰好是类,就有可能希望进一步访问它的成员。(*it).empty() it是某个容器的迭代器,用(*)解引用然后进行操作,注意()是必须的,否则‘.’运算符将由it来执行。或者it->empty(),箭头运算符直接获取容器对应的方法。
	\item \textbf{迭代器运算} iter+n 移动n步;iter += n将迭代器移动n的结果赋给iter;iter2-iter1两个迭代器之间的距离。
\end{itemize}

\begin{tcolorbox}[boxsep=-0.05in]
	迭代器中往往采用v.begin() != v.end()来判断是否为空,而非大于号小于号,这是因为迭代器中往往只定义了!=操作,更符合C++程序员的习惯。
\end{tcolorbox}

\subsubsection{数组}
与vector类似的数据结构,在性能上优于vector,但是在灵活性上差,如果事先不能确定元素的确切个数,用vector。int a[10]; int * pa[10];string str[10]; 定义的时候必须有具体维度。同时也不存在引用的数组。也不能用auto。char a4[] = "C++"; 注意双引号括起来的数组最后隐含一个空字符。
\begin{itemize}
	\item \textbf{存放指针的数组} int *ptrs[10];
	\item \textbf{数组的指针} int (*Parray)[10] =\&arr; 指向一个含有10个整数的数组;这里理解起来就是有一些特异性,先声明是个指针,再说明是一个指向包含10个元素的数组的指针。
	\item \textbf{数组的引用} int (\&Rarray)[10] = arr; 引用一个含有10个整数的数组;
	\item int *(\&arry)[10]=ptrs;arry是一个引用,引用的是一个包含10个元素的指针数组。理解数组,从数组名字开始由内到外阅读。
	\item \textbf{指针与数组} string nums[]={"one","two","three"}; string *p = nums; string *p=\&nums[0]; 指针也是迭代器。
	\item \textbf{数组的begin end} int *beg = begin(nums); 由于数组不是类类型,C++11提供了两个新函数完成这个工作。
\end{itemize}

\subsubsection{多维数组}
多维数组只是数组的数组。引用和多维数组与指针和多维数组差不多。指针 ++p,是在对应维度进行指针的递增,这也印证了多维数组只是数组的数组。
\begin{itemize}
	\item int arr[10][100][1000] = {};
	\item {范围for循环}是在C++11中新加的。
	\begin{lstlisting}[caption={}]
		size_t cnt = 0;
		for (auto &row : ia)
			for (auto &col : row){
				col = cnt;
				++cnt;
			}
	\end{lstlisting}
\end{itemize}

\subsubsection{C风格字符串}
最好别用,要用就用标准库string。cstring 是 string.h 的C++版本。

\subsubsection{与旧代码的接口}
\begin{itemize}
	\item 为了\textbf{兼容C风格字符串},C++提供了 string.c\_str成员函数,来把值付给 char *str。也就是const char *str=s.c\_str();
	\item \textbf{使用数组初始化vector对象} 不允许数组为另一个内置对象赋初值,也不允许使用vector对象初始化数组,相反的,允许使用数组来初始化vector对象。int int\_arr[] = {0,1,2,3,4};$vector<int> vi(begin(int\_arr)+1,end(int\_arr));$ 反正只要传入两个指针就可。
\end{itemize}

\subsection{表达式}
\subsubsection{基础}
\begin{itemize}
	\item 一元运算符,二元运算符
	\item \textbf{重载运算符},当运算符用于类类型的运算对象时,可以重载。
	\item \textbf{左值和右值} 当一个对象被用作右值的时候,用的是对象的值(内容);当对象被用作左值的时候,用的是对象的身份(在内存中的位置)。一个原则是当需要用到右值的地方可以用左值替代,但不能反过来,左值永远可以获取到对应的右值。
\end{itemize}

\subsubsection{算术运算符}
+ - * / \%  

\subsubsection{逻辑和关系运算符}
$! > < <= >= != == \&\& || $

\subsubsection{赋值运算符}
= 左侧必须是一个可修改的左值。赋值运算符优先级较低。

\subsubsection{递增和递减运算符}
++ --

前置版本:先做加减运算,然后将改变后的值作为求值结果。
后置版本:先有一个副本作为求值结果,在加减运算。除非必须,否则不用。

\subsubsection{成员访问运算符}
->
\subsubsection{条件运算符}
cond ? expr1 : expr2;

\subsubsection{位运算符}
bitset的标准库。内置运算符。移位运算符是其重载。
\begin{itemize}
	\item ~位求反
	\item $<<$ 左移
	\item $>>$ 右移
	\item \& 位与
	\item 位异或
	\item $|$ 位或
\end{itemize}

\subsubsection{sizeof运算符}
返回所占的字节数。

\subsubsection{隐式转换/显示转换}
$cast-name<type>(expression);$ 一般别用。

$ slope = static\_cast<double>(j) / i;$只要没有底层const就可以用。一个用法是用其找回存放在void*中的指针值。

const\_cast只用于改变运算对象的底层const。再次确认 const char * 和 char *是两个类型。const和char是一体的。

reinterpret\_cast通常位运算对象的位模式提供较低层次上的重新解释。一般不用。

\subsection{语句}

\begin{itemize}
	\item \textbf{if} if (cond) statement if else
	\item \textbf{switch} switch(ch) {case 1: break; case 2: break; default: break;}
	\item \textbf{while} 
	\item \textbf{for} 传统for和范围for。传统for的定义变量可以定义多个,但必须都是同样的类型。范围for主要用来遍历容器和其他序列的所有元素。
	\item \textbf{do while} do statement while(condition);
	\item break和continue。
	\item goto 别用。
	\item try语句和异常处理。
	
	$\#include <stdexcept>$
	
	throw表达式来抛出异常。throw runtime\_error("it is wrong");
	try+cactch捕捉异常。
	一套异常类(expression class)用于在catch和throw之间传递消息。
	\begin{lstlisting}[caption={}]
	try {
		if()
		throw runtime\_error("it is wrong"); 被第二个catch捕获。
	} catch(exception-declaration){
		handler;
	} catch(runtime_error err){
		error.what() 输出error的内容
	}
	\end{lstlisting}
	\item 具体异常类可见书P176。
	
	
\end{itemize}

\subsection{函数}
\begin{itemize}
	\item 函数定义的地方叫形参,调用叫实参。
	\item 函数调用完成两项工作:一个是用实参初始化形参,二是将控制权转移给被调用的函数。主调函数被终端,被调函数开始执行。
	\item 名字有作用域,对象有生命周期。
	\item \textbf{局部静态对象} static size\_t ctr = 0;在函数结束之后仍然存在。
	\item \textbf{函数声明} 在头文件中进行函数声明,在源文件中定义。
	\item 当初始化一个非引用类型的变量时,初始值会被拷贝给变量,但是对变量的改动不会影响初始值。
	\item 使用引用和指针,避免拷贝。使用引用形参返回额外信息。
	\item \textbf{const形参和实参} 和赋值时遵循一样的规则。为了能把const的值传进函数,对于不会改变的形参,尽可能的定义为const。
	\item \textbf{数组引用形参} 切记是 int (\&arr)[10]; 括起来表达的是一个引用,不括是引用的数组。
	\item \textbf{int main(int argc, char *argv[]);}
	\item \textbf{含有可变形参的函数} 
	
	1. 如果函数的实参数量未知,但是全部实参的类型都相同,使用initializer\_list类型的形参(一种标准库类型)。$void error\_msg(initializer\_list<string> li)$。类似vector。
	
	2. 省略符形参,这个主要是为了方便C++程序访问C代码,使用了varargs的C标准库功能。void foo(para\_list, ...);
	\item 返回类型和return语句。返回引用的时候是个左值,可以放在等号的左边。就是刚刚返回就被改变了,不会有人这样用。
	\item 主函数main可以没有返回值。
	\item 递归函数
	\item \textbf{返回数组指针} int (*func(int i))[10];
	\item 尾置返回类型 auto func(int i) -> int(*)[10];
	\item \textbf{函数重载} 函数名字相同,但是形参不同。
	\item 重载必须声明在同一级作用域上,不然就可能出错。
	\item 也可以有默认实参。多次声明中,一个形参只能有一次默认实参。
	\item \textbf{内联函数inline}。调用函数一般比求等价表达式值要慢一些,因为调用包含拷贝之类的过程。而内联函数在调用点“内联的”展开。只需在函数返回类型前面加上关键字inline。一般函数一般用于规模较小,频繁调用的函数。
	\item \textbf{constexpr函数}是指能用于常量表达式的函数,函数的返回类型以及所有的形参的类型都得是字面值。constexpr隐式的被定义为内联函数。
	\item 内联函数和constexpr函数通常定义在头文件中。
	\item \textbf{assert 和 NDEBUG预处理宏} 可以用于屏蔽调试代码。
	
	assert(expr)首先对expr求值,如果为假,则assert输出信息病终止程序。话说这个东西不是叫断言么。assert定义在cassert头文件中。
	
	NDEBUG预处理变量,如果定义了$\#define NDEBUG$则不执行assert,或者g++ -D NDEBUG也可以。
	或者$\#ifdef NDEBUG xxxx \#endif$
	
	此外,\_ \_FILE\_ \_ \_ \_LINE\_ \_ \_ \_TIME\_ \_ \_ \_DATE\_ \_分别定义了文件名,当前行号,编译时间,编译日期。
\end{itemize}

\subsubsection{函数匹配}
确定调用选择哪个重载函数。

\subsubsection{函数指针}
函数的类型由它的返回类型和形参类型共同决定,与函数名无关。

对于 bool lengthCompare(const string \&, const string \&);他的类型是

bool(const string \&, const string \&);

对应的指针是 bool (*pf)(const string \&, const string \&);其中括号必不可少,不然就是一个返回值是bool指针的函数。

pf=lengthCompare;

bool b1 = pf("Hello","goodbye");

bool b2 = *pf("Hello","goodbye");两者等价,不必解引用也可以。

重载函数必须清晰界定是哪个函数。

void useBigger(const string \&s1, const string \&s2, bool (pf)(const string\&, const string \&));

void useBigger(const string \&s1, const string \&s2, bool (*pf)(const string\&, const string \&));

\subsection{类}

数据抽象和封装,数据抽象依赖于接口和实现,封装实现了类的接口和实现的分离。

\begin{itemize}
	\item 利用public和private访问说明符,为类添加封装性。
	\item 使用关键字class定义类。struct和class都可以,唯一区别是默认访问权限不一样。在第一个访问说明符之前,struct关键字下是public的,而class的是private的。class是以private为主。
	\item \textbf{友元}。类允许其他类或者函数访问它的非公有成员的方法。在类中用friend关键字声明友元函数。
	\item 最好还是在类的外部提供一个独立的声明。
	\item 成员函数也可以被重载。
	\item \textbf{mutable}一个可变数据成员永远不会是const,即使他是const对象的成员。对于 void func() const{}函数是一个只读函数,但是对于有mutable声明的变量,也可以修改。
	\item C++11, 类内初始值。
	\item 返回*this的成员函数,return*this,就可以 myScreen.move(4,0).set("\#").
	\item 类类型,即使成员函数完全一样,不同的类就是不同的类型。
	\item 类声明。
	\item 可以把其他类定义成友元,也可以把其他类的成员函数定义为友元. friend class Windows\_mgr; friend void Window\_mgr::clear(ScreenIndex)
	\item \textbf{委托构造函数}.算是语法糖吧,没啥意思.
	\item 在实际中,如果定义了其他构造函数,最好也提供一个默认构造函数.
	\item 可以在类成员函数前添加\textbf{explicit}禁止隐式的数据类型转换.
	\item 当一个类,1. 全部的成员都是public的,2. 没有定义任何构造函数, 3. 没有类内初始值, 4.没有基类,没有virtual函数,就叫\textbf{聚合类}.可以直接初始化.
	\item \textbf{static}静态成员,静态成员存在于任何对象之外,为类的所有对象所共享. static只出现在类的内部.
\end{itemize}

\subsection{IO库}
C++不直接处理输入输出,而是定义了标准库来处理IO,这些类型支持从数据读取,向设备写入.设备可以是文件,控制台窗口.
区别于第一章进行记录.iostream定义了用于读写流的,fstream定义了读写文件,sstream读写内存string对象的类型.

ifstream和istringstream都是继承自istream的. 因此在调用上是一样的.

IO操作容易发生错误,有时候也需要知道为什么发生错误,然后选择正确的处理方式,因此iostream提供了流状态的查询.发生错误了就用setstate把对应位置置位,然后cin.eof()等函数查看到底是怎么错的,继续做.
\begin{lstlisting}[caption={}]
	auto old\_state = cin.rdstate();
	cin.clear(); 复位
	process_input(cin);
	cin.setstate(old\_state); 表示发生对应错误
	cin.clear(cin.rdstate \& ~cin.failbit \& ~cin.badbit); ~是位求反.
\end{lstlisting}

endl, flush, ends, unitbuf, nounitbuf,

打开文件可能出错,因此if(file),检查一下是对的,其实就是检查前文的rdstate,有一个错,就是错.应该是与操作.不同的错误发生相应的置位.

in读模式,out写模式,app每次写定位到文件末尾,ate打开文件后立即到文件末尾,trunc截断文件,binary以二进制方式打开文件io. 可以指定多个模式.

\textbf{String流}定义了三个类型来支持内存IO,把string当作一个IO流.

\subsection{顺序容器}
\subsubsection{简介}
顺序容器的元素与元素加入容器的位置相对应。而有序和无序关联容器则根据关键字的值来存储元素。
不同的实现在:
	1. 添加删除容器的代价,2.非顺序访问容器时的代价都有折衷。链表实现和数组实现。
\begin{itemize}
	\item vector,可变大小,在尾部之外插入或者删除可能很慢。
	\item deque,双端队列,支持快速随机访问,在头尾插入很快。
	\item list,双向链表,只支持双向顺序访问,任意位置插入都很快。
	\item forward\_list,单向链表,只单向访问,任意位置。
	\item array,固定大小,快速访问,不支持插入删除。
	\item string,与vector相似,但只原来保存字符。
	\item 迭代器是list和vector共同的操作,如果具体情况下不能确定用哪个,用迭代器访问,必要时候选择具体的数据结构。
\end{itemize}

\subsubsection{容器库概览}
提供的元素类型必须有构造函数。如何插入元素:
\begin{itemize}
	\item push\_back()和push\_front,
	\item 在容器的特定位置添加元素,insert(),对于数组类型,插入到任何位置都可以,但是很耗时。svec.insert(svec.begin(),"Hello!");
	\item 插入范围内元素,svec.insert(svec.end(),10,"Hello");返回值是插入后的首元素。
	\item \textbf{emplace()}, C++11新标准,调用emplace与insert不同,是将参数传递给元素类型的构造函数。emplace成员使用这些参数在容器管理的内存空间中直接构造函数,而不是拷贝。emplace是利用了构造函数。
	\item 赋值,assign,seq.assign(begin(),end());
	\item 访问元素,begin() end() front() back(),需要注意的是end()指向的是最后一个的后一个元素。
	\item 也可以通过下标操作,但是下标运算符需要确保不越界。svec.at(0),可以使用at函数来完成,如果越界会抛出out of range
	\item 删除首位元素,pop\_back(), pop\_front() 
	\item 从容器内部删除一个元素,erase(it),返回下一个元素;删除多个元素,slist.erase(begin(),end())
	\item forward\_list()具有自己的插入删除操作。	
	\item 改变容器大小,resize(),主要是array的方法,其他都是自动的,然后具体的添加有自己的规则。
	\item 因为动态容器会改变内存结构,存储结构被重新分配,因此容器操作会使迭代器失效。必须保证每次改变容器的操作之后都必须重新定位迭代器。
	因此只要添加删除元素,就要在每个迭代步中更新迭代器,引用或指针。更新的方式就是用添加删除函数的返回值。比如insert之后,返回的是被插入的当前值,需要向后移动两个掠过新加入的值。不要保存end()的返回值。
	\item vector对象是如何增长的。需要从旧位置移动到新空间中,然后添加新元素,释放旧存储空间。
	\item capacity()告诉我们不扩张空间的情况下可以容纳多少个元素和reserve()操作允许我们通知容器它应该准备保存多少个元素。
\end{itemize}

\subsubsection{额外的string操作}
\begin{itemize}
	\item substr操作,string s("hello world"); string s2 = s.substr(0,5);
	\item insert和erase有更多的操作
	\item append和replace,append在string末尾插入操作,relpace调用的是erase和insert的简写形式。
	\item string搜索操作,一共有六个搜索操作,每个搜索操作有四个重载版本。返回匹配到的下标。string name("AnnaBelle"); auto pos1 = name.find("Anna"); 大小写敏感的。
	还有find\_first\_of和find\_first\_not\_of,在s中查找args中任意一个字符第一次出现的位置。
	\item compare函数,比较等于,大于还是小于。
	\item 数值转换。to\_string(),将整数i转换成字符的形式;stod()将字符串s转换为浮点数。
\end{itemize}

\subsubsection{容器适配器}
顺序容器适配器。有点儿像卷积神经网络,就是以数组或者链表的数据结构,实现堆栈,队列以及优先级队列。 stack,queue,和priority\_queue。

$stack<int> stk(deq);$ deq是一个$deque<int>$,用deq来初始化一个新的stack。

默认情况下,stack和queue是deque是实现的;priority\_queue是在vector之上实现的,但是也可以在创建适配器时将一个命名的顺序容器作为第二个类型参数,来重载默认容器类型。
$stack<string,vector<string>> str\_stk;$

对于一个给定的适配器,能用哪些容器是有限制的。换句话说,其实大多数容器都可以像适配器一样用,适配器只是添加了束缚,使其行为更加贴切于某种抽象数据结构。

\subsection{泛型算法}
标准库并未给每个容器添加大量功能,而是提供了一组算法,这些算法中的大多数都独立于任何特定的容器。

\textbf{泛型} 是 \textbf{通用} 的意思。

algorithm





